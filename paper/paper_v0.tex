\documentclass{article}
\title{A monotone linearity-preserving finite volume scheme for anisotropic diffusion equation in three dimensional tetrahedral grids}
\date{TO FILL}
\author{TO FILL}
\usepackage{mathtools}
\usepackage{geometry}
\usepackage{color,soul}
\usepackage{graphicx}
\geometry{legalpaper, portrait, margin=1in}


\begin{document}
\pagenumbering{gobble}
\maketitle
\newpage
\pagenumbering{arabic}

\begin{abstract} 
A new approach for discretizing diffusive phenomena in three dimensional meshes is formulated with a stencil capable of handling highly heterogeneous and anisotropic domains. Initially, the scheme was developed for tetrahedral meshes, although the extension for general polyhedral volumes The scheme has only cell unknowns, since the auxiliary unknowns of the stencil are interpolated based on a control-volume approach. The performance of the scheme is tested through a series of benchmark diffusion problems. \hl{insert some of the results obtained}
\end{abstract}

\newpage

\section{Introduction}
\begin{description}
\item[ $\bullet$ Application and purposes for diffusion problems (physics phenomena)]
\item[ $\bullet$ Different techniques for discretizing the steady-state diffusion equation]
\item[ $\bullet$ Multipoint Flux Approximation schemes]
\item[ $\bullet$ 2D MPFAD]
\item[ $\bullet$ 3D extension (current developments)]
\item[ $\bullet$ Other methods and current state of the art]
\end{description}

This work is organized as follows: Section 2 will deal with the mathematical and physical formulation of the diffusivity problems, Section 3 will present some of the results and section 4 will highlight some of the conclusions we draw from this work as well as point out future directions.

\section{Mathematical Formulation}
	\paragraph{In this section, we describe the physical problem as well as the mathematical modeling.} 
	\subsection{Description of the diffusion problem}

General diffusive problems are characterized by the elliptic partial differential equations.
\begin{equation}
\begin{split}
- \nabla \cdot \big( K \nabla \cdot p \big) = f  \qquad in\ \Omega, \\
u = g_{D} \qquad on\ \Gamma_{D}, \\
- K \nabla p \cdot \vec{n} = g_{N} \qquad on\ \Gamma_{N}
\end{split}
\end{equation}
	
	The Diffusive coefficient, read $K$, is a symmetric full tensor:

\begin{equation}
K = \begin{bmatrix}
	k_{xx} \ k_{xy} \ k_{xz} \\
	k_{yx} \ k_{yy} \ k_{yz} \\
	k_{zx} \ k_{zy} \ k_{zz} 
\end{bmatrix}
\end{equation}
and $f$ is the source term. The domain, $\Omega$, defined in $R^3$ is bounded by Dirichlet faces, $\Gamma_{D}$, and Neumann faces, $\Gamma_{N}$, characterizing the diffusive problem.
	
	\subsection{Numerical formulation of the Multipoint flux approximation with Diamond stencil (MPFA-D)}
	
		\subsubsection{The Linearity-preserving gradient}
For a given Tetrahedron, with a volume, say $R$, we want to show that the gradient of a generic property, $p$, satisfies the Linearity-Preserving Criterion (LPC). That is, the gradient of such property can be approximated as a vector within the Tetrahedron's volume. Hence:
	\begin{equation} \label{eq:e1}											
		\begin{pmatrix}
				p_{I} - p_{R} \\ p_{K} - p_{R}  \\ p_{J} - p_{R}
			\end{pmatrix}
		\approx
		\begin{pmatrix}
			\vec{RI}^{T} \\ \vec{RK}^{T} \\ \vec{RJ}^{T}
		\end{pmatrix}\cdot \nabla p \approx X \cdot \nabla p		
	\end{equation}
	and then:

	\begin{equation} \label{eq:e2}													
		\nabla p 
		\approx X^{-1}
			\begin{pmatrix}
				p_{I}- p_{R} \\ p_{K} - p_{R} \\ p_{J} - p_{R} 
			\end{pmatrix}
	\end{equation}
where:
	\begin{displaymath}	
	X =
	\begin{bmatrix}
	I_{i} - R_{i} & I_{j} - R_{j} & I_{k} - R_{k} \\
	K_{i} - R_{i} & K_{j} - R_{j} & K_{k} - R_{k} \\
	J_{i} - R_{i} & J_{j} - R_{j} & J_{k} - R_{k}
	\end{bmatrix}
	\end{displaymath}

The inverse of $X$ is given by:
	\begin{equation}	\label{eq:e3}
		X^{-1} = \frac{1}{\det{X}}
		\begin{bmatrix}
			\begin{vmatrix}
				K_{j} - R_{j} & K_{k} - R_{k} \\
				J_{j} - R_{j} & J_{k} - R_{k}
			\end{vmatrix} &
			\begin{vmatrix}
				K_{k} - R_{k} & K_{j} - R_{j} \\
				I_{k} - R_{k} & I_{j} - R_{j}
			\end{vmatrix} &
			\begin{vmatrix}
				J_{j} - R_{j} & J_{k} - R_{k} \\
				I_{j} - R_{j} & I_{k} - R_{k}
			\end{vmatrix}
		\\
			\begin{vmatrix}
				K_{k} - R_{k} & K_{i} - R_{i}  \\
				J_{k} - R_{k} & J_{i} - R_{i}
			\end{vmatrix} &
			\begin{vmatrix}
				K_{i} - R_{i} & K_{k} - R_{k}  \\
				I_{i} - R_{i} & I_{k} - R_{k}
			\end{vmatrix} &
			\begin{vmatrix}
				J_{k} - R_{k} & J_{i} - R_{i}  \\
				I_{k} - R_{k} & I_{i} - R_{i} 
			\end{vmatrix}
		\\
			\begin{vmatrix}
				K_{i} - R_{i} & K_{j} - R_{j}  \\
			    J_{i} - R_{i} & J_{j} - R_{j}
			\end{vmatrix} &
			\begin{vmatrix}
				K_{j} - R_{j} & K_{i} - R_{i} \\
				I_{j} - R_{j} & I_{i} - R_{i}
			\end{vmatrix} &
			\begin{vmatrix}
				J_{i} - R_{i} & J_{j} - R_{j} \\
				I_{i} - R_{i} & I_{j} - R_{j} 
			\end{vmatrix}
		\end{bmatrix}		
	\end{equation}

Since the determinant of $X$ is determined by the triple product property of the vectors $\vec{IR}$, $\vec{JR}$ and $\vec{KR}$. That is:
	\begin{displaymath}
	\det{X} = |\vec{RI}\cdot(\vec{RK} \times \vec{RJ})| = 6V_{R}
	\end{displaymath}

Therefore, eq. \ref{eq:e3} can be rearranged to yield:
	\begin{equation} \label{eq:e4} 
		\nabla u \approx \frac{1}{6V_{R}}
		\begin{bmatrix}
			\begin{vmatrix}
				K_{j} - R_{j} & K_{k} - R_{k} \\
				J_{j} - R_{j} & J_{k} - R_{k}
			\end{vmatrix} &
			\begin{vmatrix}
				K_{k} - R_{k} & K_{j} - R_{j} \\
				I_{k} - R_{k} & I_{j} - R_{j}
			\end{vmatrix} &
			\begin{vmatrix}
				J_{j} - R_{j} & J_{k} - R_{k} \\
				I_{j} - R_{j} & I_{k} - R_{k}
			\end{vmatrix}
		\\
			\begin{vmatrix}
				K_{k} - R_{k} & K_{i} - R_{i}  \\
				J_{k} - R_{k} & J_{i} - R_{i}
			\end{vmatrix} &
			\begin{vmatrix}
				K_{i} - R_{i} & K_{k} - R_{k}  \\
				I_{i} - R_{i} & I_{k} - R_{k}
			\end{vmatrix} &
			\begin{vmatrix}
				J_{k} - R_{k} & J_{i} - R_{i}  \\
				I_{k} - R_{k} & I_{i} - R_{i} 
			\end{vmatrix}
		\\
			\begin{vmatrix}
				K_{i} - R_{i} & K_{j} - R_{j}  \\
			    J_{i} - R_{i} & J_{j} - R_{j}
			\end{vmatrix} &
			\begin{vmatrix}
				K_{j} - R_{j} & K_{i} - R_{i} \\
				I_{j} - R_{j} & I_{i} - R_{i}
			\end{vmatrix} &
			\begin{vmatrix}
				J_{i} - R_{i} & J_{j} - R_{j} \\
				I_{i} - R_{i} & I_{j} - R_{j} 
			\end{vmatrix}
		\end{bmatrix}
		\begin{pmatrix}
				p_{I} - p_{R} \\ p_{K} - p_{R}  \\ p_{J} - p_{R}
			\end{pmatrix}
	\end{equation}
or:
	\begin{equation} \label{eq:e5}
		\nabla u \approx \frac{1}{6V_{R}}
		\begin{bmatrix}
		(p_{I} - p_{R})
			\begin{pmatrix}
				\begin{vmatrix}
					K_{j} - R_{j} & K_{k} - R_{k} \\
					J_{j} - R_{j} & J_{k} - R_{k}
				\end{vmatrix} &
				\begin{vmatrix}
					K_{k} - R_{k} & K_{j} - R_{j} \\
					I_{k} - R_{k} & I_{j} - R_{j}
				\end{vmatrix} &
				\begin{vmatrix}
					J_{j} - R_{j} & J_{k} - R_{k} \\
					I_{j} - R_{j} & I_{k} - R_{k}
				\end{vmatrix}
			\end{pmatrix}
		\\
		+ (p_{K} - p_{R})
			\begin{pmatrix}
				\begin{vmatrix}
					K_{k} - R_{k} & K_{i} - R_{i} \\
					J_{k} - R_{k} & J_{i} - R_{i}
				\end{vmatrix} &
				\begin{vmatrix}
					K_{i} - R_{i} & K_{k} - R_{k} \\
					I_{i} - R_{i} & I_{k} - R_{k}
				\end{vmatrix} &
				\begin{vmatrix}
					R_{k} - J_{k} & J_{i} - R_{i} \\
					I_{k} - R_{k} & I_{i} - R_{i}
				\end{vmatrix}
			\end{pmatrix}
		\\
		+ (p_{J} - p_{R})
			\begin{pmatrix}
				\begin{vmatrix}
					K_{i} - R_{i} & K_{j} - R_{j} \\
					J_{i} - R_{i} & J_{j} - R_{j}
				\end{vmatrix} &
				\begin{vmatrix}
					K_{j} - R_{j} & K_{i} - R_{i}\\
					I_{j} - R_{j} & I_{i} - R_{i}
				\end{vmatrix} &
				\begin{vmatrix}
					J_{i} - R_{i} & J_{j} - R_{j} \\
					I_{i} - R_{i} & I_{j} - R_{j}
				\end{vmatrix}
			\end{pmatrix}
		\end{bmatrix}
	\end{equation}

Note that in eq. \ref{eq:e5}, the product between the potential difference of property $p$ along different directions by the the two-by-two determinant terms form a normal vector to one specific side of the tetrahedron pointing outward with a modulus two times the area of the triangular face (or the area of the parallelogram formed by the vectors). Thus, we have:

	\begin{equation} \label{eq:e6} 													
		\nabla u \approx \frac{1}{6V_{R}}
		\begin{bmatrix}
		(p_{I} - p_{R})(\vec{RK} \times \vec{RJ}) \\
		+
		(p_{K} - p_{R})(\vec{RJ} \times \vec{RI}) \\	
		+
		(p_{J} - p_{R})(\vec{RI} \times \vec{RK})
		\end{bmatrix}	
	\end{equation}
and thus:
	\begin{equation}	\label{eq:gradiente_direita}												
			\nabla u \approx \frac{1}{3V_{R}} \Big[
		(p_{I} - p_{R}) \cdot \vec{N}_{RKJ} +
		(p_{K} - p_{R})\cdot \vec{N}_{RJI}	+ 
		(p_{J} - p_{R})\cdot \vec{N}_{RIK}
		 \Big]
	\end{equation}
	Applying the Gauss theorem over the tetrahedral volume, we have:

\begin{equation} \label{eq:Gauss}
\vec{N_{RIK}} = -\vec{N}_{RKJ} - \vec{N}_{RJI} - \vec{N}_{IJK}
\end{equation}
Replacing \ref{eq:Gauss} in \ref{eq:gradiente_direita}, we have:

\begin{equation} \label{eq:gradiente_direita_Gauss}
\nabla p_{R} \approx \frac{1}{3V_{R}}\bigg[ \big( p_{I} - p_{R}\big) \cdot \vec{N}_{RKJ} + 
\big( p_{J} - p_{R}\big) \cdot \bigg(-\vec{N}_{RKJ} - \vec{N}_{RJI} - \vec{N}_{IJK} \bigg) +
\big( p_{K} - p_{R}\big) \cdot \vec{N}_{RJI} \bigg]
\end{equation}

or:

\begin{equation} \label{eq:gradiente_direita_vetores_rearranjado}
\nabla p_{R} \approx \frac{1}{3V_{R}}\bigg[ \big( p_{I} - p_{J}\big) \cdot \vec{N}_{RKJ} + 
\big( p_{R} - p_{J}\big) \cdot \vec{N}_{IJK} +
\big( p_{K} - p_{J} \big) \cdot \vec{N}_{RJI} \bigg]
\end{equation}

$\vec{N_{RKJ}}$ and $\vec{N_{RJI}}$ can be replaced by, respectively:
\begin{equation} \label{eq:N_RKJ_vec}
2 \vec{N}_{RKJ} = \vec{JR} \times \vec{JK}
\end{equation}
\begin{equation} \label{eq:N_RJI_vec}
2 \vec{N}_{RJI} = \vec{JI} \times \vec{JR} 
\end{equation}

Thus, \ref{eq:gradiente_direita_vetores_rearranjado}, yields: 
\begin{equation} \label{eq:gradiente_direita_vetores_rearranjado_N_IJK}
\nabla p_{R} \approx \frac{1}{6V_{R}}\bigg[ \big( p_{I} - p_{J}\big) \cdot (\vec{JR} \times \vec{JK}) + 
2\big(p_{R} - p_{J}\big) \cdot \vec{N}_{IJK} +
\big( p_{K} - p_{J} \big) \cdot (\vec{JI} \times \vec{JR}) \bigg]
\end{equation}
Note that $\vec{JR}$ can be rewritten as a combination of $\vec{JI}$, $\vec{JK}$ e $\vec{N_{IJK}}$. As long as the latter forms a non-coplanar basis, we have the following identity:

\begin{equation} \label{eq:vetor_JR}
\vec{JR} = m\vec{JI} + n\vec{JK} + w\vec{N}_{IJK}
\end{equation}

with:

\begin{equation} \label{eq:const_m}
m = \frac{\langle(\vec{JK} \times \vec{N}_{IJK}), \vec{JR}\rangle}{2 |\vec{N}_{IJK}|^2} 
\end{equation}

\begin{equation} \label{eq:const_n}
n = \frac{\langle(\vec{N}_{IJK} \times \vec{JI}), \vec{JR}\rangle}{2 |\vec{N}_{IJK}|^2}
\end{equation}

\begin{equation} \label{eq:const_w}
w = \frac{h_{R}}{|\vec{N}_{IJK}|}
\end{equation}

performing the proper substitution of \ref{eq:vetor_JR} - \ref{eq:const_w} in \ref{eq:gradiente_direita_vetores_rearranjado_N_IJK}, one gets:

\begin{equation} \label{eq:gradiente_direita_vetores_rearranjado_N_IJK_RJ_mnw}
\begin{split}
\nabla p_{R} \approx \frac{1}{6V_{R}} \Bigg[ \big( p_{I} - p_{J}\big) \cdot  \Bigg( \frac{\langle(\vec{JK} \times \vec{N}_{IJK}), \vec{JR}\rangle}{2 |\vec{N}_{IJK}|^2}\vec{JI} \times \vec{JK}+ \frac{h_{R}}{|\vec{N}_{IJK}|}\vec{N}_{IJK}\times \vec{JK}\Bigg)+ \\
\big( p_{K} - p_{J} \big) \cdot \Bigg(\frac{\langle(\vec{N}_{IJK} \times \vec{JI}), \vec{JR}\rangle}{2 |\vec{N}_{IJK}|^2} \vec{JI} \times \vec{JK} +  \frac{h_{R}}{|\vec{N}_{IJK}|}\vec{JI} \times \vec{N}_{IJK}\Bigg) + 2 \big(p_{J} - p_{R}\big) \cdot \vec{N}_{IJK} \Bigg]
\end{split}
\end{equation}
 \ref{eq:gradiente_direita_vetores_rearranjado_N_IJK_RJ_mnw} similarly satisfies the relation with the area vector $-2\vec{N_{IJK}}$ and the vector product $\vec{JI} \times \vec{JK}$ such as in \ref{eq:gradiente_direita_vetores_rearranjado_N_IJK}:

\begin{equation} \label{eq:gradiente_direita_final}
\begin{split}
\nabla p_{R} \approx \frac{1}{6V_{R}} \Bigg[ \big( p_{J} - p_{I}\big) \cdot \Bigg( \frac{\langle(\vec{JK} \times \vec{N}_{IJK}), \vec{JR}\rangle}{|\vec{N}_{IJK}|^2}\vec{N}_{IJK} - \frac{h_{R}}{|\vec{N}_{IJK}|} \Big(\vec{N}_{IJK}\times \vec{JK} \Big) \Bigg) + \\
\big( p_{J} - p_{K} \big) \cdot \Bigg(\frac{\langle(\vec{N}_{IJK} \times \vec{JI}), \vec{JR}\rangle}{|\vec{N}_{IJK}|^2} \vec{N}_{IJK} - \frac{h_{R}}{|\vec{N}_{IJK}|}\Big( \vec{JI} \times \vec{N}_{IJK}\Big) \Bigg) + 2 \big(p_{J} - p_{R}\big) \cdot \vec{N}_{IJK} \Bigg]
\end{split}
\end{equation}

In a similar manner, the left volume gradient can be determined, following the same steps adopted in the previous equations:

\begin{equation} \label{eq:gradiente_esquerda}
\nabla p_{L} \approx \frac{1}{3V_{L}}\bigg[ \big( p_{I} - p_{L}\big) \cdot \vec{N}_{LJK} + 
\big( p_{J} - p_{L}\big) \cdot \vec{N}_{LKI} +
\big( p_{K} - p_{L}\big) \cdot \vec{N}_{LIJ} \bigg]
\end{equation}

with:

\begin{equation} \label{eq:N_LKI_vec}
2 \vec{N}_{LKI} = \vec{N}_{IJK} - \vec{N}_{LJK} - \vec{N}_{LIJ}
\end{equation}

The vector $\vec{N}_{IJK}$ appears due to the cross product of  $\vec{JK}$ e $\vec{JI}$ just as for the right sided volume, hence, \ref{eq:gradiente_esquerda} yields:

\begin{equation} \label{eq:gradiente_esquerda_vetores_N_IJK}
\nabla p_{L} \approx \frac{1}{6V_{L}}\bigg[ \big(p_{I} - p_{J}\big) \cdot (\vec{LJ} \times \vec{JK}) + 
2 \big( p_{J} - p_{L}\big) \cdot \vec{N}_{IJK} + \big( p_{K} -  p_{J}\big) \cdot (\vec{JI} \times \vec{LJ}) \bigg]
\end{equation}

$\vec{LJ}$, similar to $\vec{JR}$ can be rewritten, by the same basis $\vec{JI}$, $\vec{JK}$ e $\vec{N}_{IJK}$:

\begin{equation} \label{eq:vetor_LJ}
\vec{LJ} = m'\vec{JI} + n'\vec{JK} + w'\vec{N}_{IJK}
\end{equation}

where:

\begin{equation} \label{eq:const_m_prim}
m' = \frac{\langle(\vec{JK} \times \vec{N}_{IJK}), \vec{LJ}\rangle}{2 |\vec{N}_{IJK}|^2} 
\end{equation}

\begin{equation} \label{eq:const_n_prim}
n' = \frac{\langle(\vec{N}_{IJK} \times \vec{JI}), \vec{LJ}\rangle}{2 |\vec{N}_{IJK}|^2}
\end{equation}

\begin{equation} \label{eq:const_w_prim}
w' = \frac{h_{L}}{|\vec{N}_{IJK}|} 
\end{equation}

Hence, replacing \ref{eq:vetor_LJ} - \ref{eq:const_w_prim} in \ref{eq:gradiente_esquerda_vetores_N_IJK}, we have:

\begin{equation} \label{eq:gradiente_esquerda_final}
\begin{split}
\nabla p_{L} \approx \frac{1}{6V_{L}}\Bigg\{ \big(p_{J} - p_{I}\big) \cdot \Bigg(\frac{\langle(\vec{JK} \times \vec{N}_{IJK}), \vec{LJ}\rangle}{|\vec{N}_{IJK}|^2}\vec{N}_{IJK} - \frac{h_{L}}{|\vec{N}_{IJK}|} \Big( \vec{N}_{IJK} \times \vec{JK} \Big) \Bigg)  + \\
 \big( p_{J} -  p_{K}\big) \cdot   \Bigg( \frac{\langle(\vec{N}_{IJK} \times \vec{JI}), \vec{LJ}\rangle}{|\vec{N}_{IJK}|^2}\vec{N}_{IJK} - \frac{h_{L}}{|\vec{N}_{IJK}|} \Big( \vec{JI} \times \vec{N}_{IJK} \Big) \Bigg) + 2\big( p_{J} - p_{L}\big) \cdot  \vec{N}_{IJK}\Bigg\}
\end{split}
\end{equation}
		\subsubsection{Construction of the unique flux}
The flux can be calculated by:
\begin{equation} \label{eq:darcy}
\vec{F} = - K \big( \nabla \cdot p \big)
\end{equation}
Th right volume then is obtained by replacing \ref{eq:gradiente_direita_final} in \ref{eq:darcy}:
\begin{equation} \label{eq:fluxo_direita}
\begin{split}
\int{\vec{v}_{R} \cdot \vec{n}_{IJK}dS} 
 \approx -\frac{K_{R}}{6V_{R}} \Bigg[ \big( p_{J} - p_{I}\big) \cdot  \Bigg( \frac{\langle(\vec{JK} \times \vec{N}_{IJK}), \vec{JR}\rangle}{|\vec{N}_{IJK}| ^2} \vec{N}_{IJK} - \frac{h_{R}}{|\vec{N}_{IJK}|} \Big(\vec{N}_{IJK}\times \vec{JK} \Big)\Bigg)  + \\
\big( p_{J} - p_{K} \big) \cdot  \Bigg(\frac{\langle(\vec{N}_{IJK} \times \vec{JI}), \vec{JR}\rangle}{|\vec{N}_{IJK}| ^ 2}\vec{N}_{IJK} - \frac{h_{R}}{|\vec{N}_{IJK}|} \Big( \vec{JI} \times \vec{N}_{IJK}\Big) \Bigg) + 2 \big(p_{J} - p_{R}\big) \cdot \vec{N}_{IJK} \Bigg] \cdot \vec{N}_{IJK}
\end{split}
\end{equation}

Rearranging \ref{eq:fluxo_direita}, we have:
\begin{equation} \label{eq:fluxo_direita_tensor}
\begin{split}
\int{\vec{v}_{R} \cdot \vec{n}_{IJK}dS} 
 \approx -\Bigg[ \big( p_{J} - p_{I}\big) \cdot \Bigg(\frac{\langle(\vec{JK} \times \vec{N}_{IJK}), \vec{JR}\rangle}{|\vec{N}_{IJK}|^2}\vec{N}_{IJK} ^T \frac{K_{R}}{6V_{R}} \vec{N}_{IJK} - \frac{h_{R}}{|\vec{N}_{IJK}|}\vec{N}_{IJK} ^T\frac{K_{R}}{6V_{R}} \Big( \vec{N}_{IJK}\times \vec{JK} \Big) \Bigg) + \\
\big( p_{J} - p_{K} \big) \cdot  \Bigg(\frac{\langle(\vec{N}_{IJK} \times \vec{JI}), \vec{JR}\rangle}{|\vec{N}_{IJK}| ^ 2} \vec{N}_{IJK} ^T \frac{K_{R}}{6V_{R}}\vec{N}_{IJK} - \frac{h_{R}}{|\vec{N}_{IJK}|}\vec{N}_{IJK} ^T\frac{K_{R}}{6V_{R}} \Big( \vec{JI} \times \vec{N}_{IJK} \Big) \Bigg)  + 2 \big(p_{J} - p_{R}\big) \vec{N}_{IJK} ^T\frac{K_{R}}{6V_{R}}\vec{N}_{IJK} \Bigg]
\end{split}
\end{equation}

We adopt the following notation for simplification purposes:

\begin{equation} \label{eq:fluxo_direita_tensor_simplificado}
\begin{split}
\int{\vec{v}_{R} \cdot \vec{n}_{IJK}dS} 
 \approx -\Bigg[ \big( p_{J} - p_{I}\big) \Bigg( \frac{\langle(\vec{JK} \times \vec{N}_{IJK}), \vec{JR}\rangle}{|\vec{N}_{IJK}| ^2} \frac{K^n_{R}}{h_{R}} - \frac{1}{|\vec{N}_{IJK}|}K^{JK}_{R}\Bigg) + \\
\big( p_{J} - p_{K} \big) \Bigg(\frac{\langle(\vec{N}_{IJK} \times \vec{JI}), \vec{JR}\rangle}{|\vec{N}_{IJK}| ^ 2}\frac{K^n_{R}}{h_{R}} - \frac{1}{|\vec{N}_{IJK}|}K^{JI}_{R}\Bigg)  + 
2 \big(p_{J} - p_{R}\big) \frac{K^n_{R}}{h_{R}} \Bigg]
\end{split}
\end{equation}

where:
\begin{displaymath}
	K^{n}_{R} = \Bigg( \frac{\vec{N}_{IJK}^{T} K_{R} \vec{N}_{IJK}}{|\vec{N}_{IJK}|} \Bigg)
	\end{displaymath}
	
\begin{displaymath}
		K^{{JK}}_{R} = \Bigg( \frac{\vec{N}_{IJK}^{T}K_{R} \Big( \vec{N_{IJK}} \times \vec{JK}\Big) }{|\vec{N}_{IJK}|} \Bigg)
\end{displaymath}

\begin{displaymath}
		K^{{JI}}_{R} = \Bigg( \frac{\vec{N}_{IJK}^{T}K_{R} \Big( \vec{JI} \times \vec{N}_{IJK} \Big) }{|\vec{N}_{IJK}|} \Bigg)
\end{displaymath}
	
\ref{eq:fluxo_direita_tensor_simplificado} can be modified by multiplying it by $\frac{h_{R}}{K^n_{R}}$:

\begin{equation} \label{eq:fluxo_direita_final}
\begin{split}
\frac{h_{R}}{K^n_{R}}\int{\vec{v}_{R} \cdot \vec{n}_{IJK}dS} 
 \approx -\Bigg[ \big( p_{J} - p_{I}\big) \Bigg(\frac{\langle(\vec{JK} \times \vec{N}_{IJK}), \vec{JR}\rangle}{|\vec{N}_{IJK}| ^2}  - \frac{h_{R}}{|\vec{N}_{IJK}|}\frac{K^{JK}_{R}}{K^{n}_{R}}\Bigg) + \\
\big( p_{J} - p_{K} \big) \Bigg(\frac{\langle(\vec{N}_{IJK} \times \vec{JI}), \vec{JR}\rangle}{|\vec{N}_{IJK}| ^ 2} - \frac{h_{R}}{|\vec{N}_{IJK}|}\frac{K^{JI}_{R}}{K^n_{R}}\Bigg)  + 
2 \big(p_{J} - p_{R}\big) \Bigg]
\end{split}
\end{equation}

Similarly, we can do for the left-sided volume:

\begin{equation} \label{eq:fluxo_esquerda}
\begin{split}
\frac{h_{L}}{K^n_{L}}\int{\vec{v}_{L} \cdot \vec{n_{IKJ}}dS} \approx -\Bigg[ \big(p_{J} - p_{I}\big) \Bigg(\frac{\langle(\vec{JK} \times \vec{N_{IJK}}), \vec{LJ}\rangle}{|\vec{N_{IJK}}| ^2}  - \frac{h_{L}}{|\vec{N_{IJK}}|}\frac{K^{JK}_{L}}{K^n_{L}}\Bigg)  + \\
 \big( p_{J} -  p_{K}\big) \Bigg(  \frac{\langle(\vec{N_{IJK}} \times \vec{JI}), \vec{LJ}\rangle}{|\vec{N_{IJK}}| ^2} - \frac{h_{L}}{|\vec{N_{IJK}}|} \frac{K^{JI}_{L}}{K^n_{L}} \Bigg) + 2\big( p_{J} - p_{L}\big)\Bigg]
\end{split}
\end{equation}

By applying the mass conservation law, we have that the total mass flux for the left-sided and the right-sided volumes sharing the face $N_{IJK}$ is:

\begin{equation} \label{eq:conservacao}
\int{\vec{v}_{R} \cdot \vec{n}_{IJK}dS} +\int{\vec{v}_{L} \cdot \vec{n_{IKJ}}dS} = 0
\end{equation}

then replacing \ref{eq:conservacao} in \ref{eq:fluxo_esquerda}:

\begin{equation} \label{eq:fluxo_esquerda_substituicao}
\begin{split}
\frac{h_{L}}{K^n_{L}}\int{\vec{v}_{R} \cdot \vec{n_{IJK}}dS} \approx \Bigg[ \big(p_{J} - p_{I}\big) \Bigg(\frac{\langle(\vec{JK} \times \vec{N_{IJK}}), \vec{LJ}\rangle}{|\vec{N_{IJK}}| ^2} - \frac{h_{L}}{|\vec{N_{IJK}}|}\frac{K^{JK}_{L}}{K^n_{L}}\Bigg)  + \\
 \big( p_{J} -  p_{K}\big) \Bigg( \frac{\langle(\vec{N_{IJK}} \times \vec{JI}), \vec{LJ}\rangle}{|\vec{N_{IJK}}| ^2} - \frac{h_{L}}{|\vec{N_{IJK}}|} \frac{K^{JI}_{L}}{K^n_{L}} \Bigg) + 2\big( p_{J} - p_{L}\big)\Bigg]
\end{split}
\end{equation}

by taking the average from \ref{eq:fluxo_direita_final} and \ref{eq:fluxo_esquerda_substituicao}, we have:

\begin{equation} \label{eq:fluxo_unico_termos_completos}
\begin{split}
\Bigg(\frac{h_{L}}{K^n_{L}} + \frac{h_{R}}{K^n_{R}} \Bigg)\int{\vec{v}_{R} \cdot \vec{n}_{IJK}dS} 
 \approx -\Bigg[ \big( p_{J} - p_{I} \big) \Bigg( \frac{\langle(\vec{JK} \times \vec{N_{IJK}}), \vec{LR}\rangle}{|\vec{N_{IJK}}| ^2} - \frac{h_{L}}{|\vec{N_{IJK}}|}\frac{K^{JK}_{L}}{K^n_{L}} - \frac{h_{R}}{|\vec{N_{IJK}}|}\frac{K^{JK}_{R}}{K^{n}_{R}}\Bigg) + \\
\big( p_{J} - p_{K} \big) \Bigg(\frac{\langle(\vec{N_{IJK}} \times \vec{JI}), \vec{LR}\rangle}{|\vec{N_{IJK}}| ^2}  - \frac{h_{L}}{|\vec{N_{IJK}}|} \frac{K^{JI}_{L}}{K^n_{L}} - \frac{h_{R}}{|\vec{N_{IJK}}|}\frac{K^{JI}_{R}}{K^n_{R}}\Bigg) + 2\big( p_{R} - p_{L}\big)\Bigg]
\end{split}
\end{equation}

Then, \ref{eq:fluxo_unico_termos_completos} yields:

\begin{equation} \label{eq:fluxo_unico_final}
\begin{split}
\int{\vec{v}_{R} \cdot \vec{n}_{IJK}dS} 
 \approx -K^n_{eff}\Bigg[ 2 \big(p_{R} - p_{L}\big) -D_{JI}\big( p_{I} - p_{J}\big) - D_{JK}\big( p_{K} - p_{J}\big)\Bigg]
\end{split}
\end{equation}

were:

\begin{equation}
K^n_{eff} = \frac{K^n_{R}K^n_{L}}{K^n_{R}h_{L} + K^n_{L}h_{R}}
\end{equation}
\begin{equation}
D_{JI} = \frac{\langle(\vec{JK} \times \vec{N_{IJK}}), \vec{LR}\rangle}{|\vec{N_{IJK}}| ^2} - \frac{1}{|\vec{N_{IJK}}|}\Bigg(\frac{K^{JK}_{R}}{K^{n}_{R}}h_{R}+\frac{K^{JK}_{L}}{K^n_{L}}h_{L} \Bigg)
\end{equation}
\begin{equation}
D_{JK} = \frac{\langle(\vec{N_{IJK}} \times \vec{JI}), \vec{LR}\rangle}{|\vec{N_{IJK}}| ^2}  - \frac{1}{|\vec{N_{IJK}}|}\Bigg(\frac{K^{JI}_{R}}{K^{n}_{R}}h_{R}+\frac{K^{JI}_{L}}{K^n_{L}}h_{L} \Bigg)
\end{equation}
		\subsubsection{Boundary conditions}

Dirichlet boundaries can be calculated from \ref{eq:fluxo_direita_tensor_simplificado}, which simplifying yields:

\begin{equation} \label{eq:fluxo_direita_Dirichlet}
\int{\vec{v}_{R} \cdot \vec{n}_{IJK}dS} 
 \approx -\Bigg[ 2 \frac{K^n_{R}}{h_{R}}\big(p_{R} - g^D_{J}\big)+ D_{JI}\big( g^D_{J} - g^D_{I}\big)  + 
D_{JK}\big( g^D_{J} - g^D_{K} \big)\Bigg]
\end{equation}
where $g^D_{I}$, $g^D_{J}$ e $g^D_{K}$ 
are prescribed values defined in the boundaries:
\begin{displaymath}
D_{JI} = \frac{\langle(\vec{JK} \times \vec{N_{IJK}}), \vec{JR}\rangle}{|\vec{N_{IJK}}| ^2}\frac{K^n_{R}}{h_{R}} - \frac{1}{|\vec{N_{IJK}}|}K^{JK}_{R}
\end{displaymath}

\begin{displaymath}
D_{JK} =  \frac{\langle(\vec{N_{IJK}} \times \vec{JI}), \vec{JR}\rangle}{|\vec{N_{IJK}}| ^2}\frac{K^n_{R}}{h_{R}}  - \frac{1}{|\vec{N_{IJK}}|}K^{JI}_{R}
\end{displaymath}
Also, Neumann boundary conditions can be calculated:
\begin{equation} \label{eq:neumann_bc}
\int{\vec{v}_{R} \cdot \vec{n}_{IJK}dS} = g_{N}
\end{equation}

		\subsubsection{Approximating the vertex unknowns - the LPEW2 in three dimensions}

The proposed weighting method LPEW2 for eliminating the vertex unknowns was first formulated by GAO and WU (2010) for general 2-D meshes. This section addresses the extension of the LPEW2 for three dimensional meshes. \hl{Mention here the work of contreras (in the context that our research team has already validated that LPEW2 outperforms LPEW1)} and hence, we only derive the expression for the LPEW2.
 
We want to find a relation that gives the value of the pressure on the vertex as a weighted summation of all cell centres surrounding that vertex. Therefore, a control volume is constructed by connecting all the volumes that share the same vertex to yield the following expression:
\begin{equation} \label{eq:summ_all_omega}
p_{Q}=\frac{\sum_{n=1}^{N}\omega_{n}p_{n}}{\sum_{n=1}^{N}\omega_{n}}
\end{equation}

Then, the mass conservation law can be applied to that control volume:

\begin{equation} \label{eq:summ_all_flux_around_vertex}
\sum_{i = 1} ^{N_{Q}} \big( F_{i} \cdot \vec{N}_{i} \big) \approx 0
\end{equation}
where $N_{Q}$ is the total number of cells surrounding the vertex.

\hl{There are several ways to construct that control volume around vertex $ Q $. One way, for instance explored here, is to connect all the semi-edges of the neighboring volumes. The semi-edge point to be chosen here, for simplicity, is half the length of the entire edge, although GAO and WU (2010) suggested that any values can be used.
This polyhedral shape is to contain as many tetrahedrons as there are volumes sharing the vertex $ Q $.}

Suppose that vertex $ Q $ is shared by 8 cells, and that cell $ M $  shares sides with cells $L $, $ R $ and $ W $. Then the flux outward the cell $ M $ is given through face $ ijk $ and it is given by:
\begin{equation} \label{eq:flux_m}
\vec{F}_{M} \cdot \vec{N}_{ijk} = - \frac{K_{M}}{3V_{M}} \bigg[(p_{k} - p_{Q}) \cdot \vec{N}_{jQi} + (p_{j} - p_{Q}) \cdot \vec{N}_{iQk} + (p_{i} - p_{Q}) \cdot \vec{N}_{kQj} \bigg] \cdot \vec{N}_{ijk}
\end{equation}

Note that eq. \ref{eq:flux_m} has not only showed the vertex pressure unknown to be eliminated as well as introduced the pressure unknowns to the semi-edges auxiliary points, that should also vanish. Thus we calculate the flux from $ M $ to all other neighbor volumes within the control volume $ Q $. For instance, the flux over the face shared by volumes $ M $ and $ L $ is:
\begin{equation} \label{eq:side_flux_M_L}
\vec{F}_{M|L} \cdot \vec{N}_{Qki} = - \frac{K_{M}}{3V_{M'}} \bigg[(p_{k} - p_{Q}) \cdot \vec{N}_{MQi} + (p_{i} - p_{Q}) \cdot \vec{N}_{kQM} + (p_{M} - p_{Q}) \cdot \vec{N}_{Qki} \bigg] \cdot \vec{N}_{Qki}
\end{equation}

where $ V_{M'} $ is the volume formed by the tetrahedron $ MkiQ $ as shown in \hl{figure to be inserted with the side flux tetrahedrons}. Rearranging eq. \ref{eq:side_flux_M_L}:
\begin{equation} \label{eq:side_flux_M_L_rearranged}
\vec{F}_{M|L} \cdot \vec{N}_{Qki} = -\xi_{M'L,k}(p_{k} - p_{Q}) - \xi_{M'L,i}(p_{i} - p_{Q}) - \xi_{M'L,M}(p_{M} - p_{Q})
\end{equation}
where:
\begin{displaymath}
	\xi_{M'L,k} = \frac{\vec{N}^{T}_{Qki} K_{M} \vec{N}_{MQi}}{3V_{M'}}
\end{displaymath}

\begin{displaymath}
	\xi_{M'L,i} = \frac{\vec{N}^{T}_{Qki} K_{M} \vec{N}_{kQM}}{3V_{M'}}
\end{displaymath}

\begin{displaymath}
	\xi_{M'L,M} = \frac{\vec{N}^{T}_{Qki} K_{M} \vec{N}_{iQk}}{3V_{M'}}
\end{displaymath}

Similarly, the flux from $L$ to $M$ can be defined as:
\begin{equation} \label{eq:side_flux_L_M_rearranged}
\vec{F}_{L|M} \cdot (-\vec{N}_{Qki}) = -\xi_{L'M,k}(p_{k} - p_{Q}) - \xi_{L'M,i}(p_{i} - p_{Q}) - \xi_{L'M,L}(p_{L} - p_{Q})
\end{equation}

By the mass conservation law:

\begin{equation} \label{eq:side_flux_L_M_summation}
\vec{F}_{L|M} \cdot (-\vec{N}_{Qki}) + \vec{F}_{M|L} \cdot \vec{N}_{Qki}= 0
\end{equation}
the summation of eqs. \ref{eq:side_flux_L_M_rearranged} and \ref{eq:side_flux_M_L_rearranged} yield:
\begin{equation} \label{eq:side_flux_M_L_summation_rearranged}
\begin{split}
-\xi_{M'L,k}(p_{k} - p_{Q}) - \xi_{M'L,i}(p_{i} - p_{Q}) - \xi_{M'L,M}(p_{M} - p_{Q}) \\ - \xi_{L'M,k}(p_{k} - p_{Q}) - \xi_{L'M,i}(p_{i} - p_{Q}) - \xi_{L'M,L}(p_{L} - p_{Q}) = 0
\end{split}
\end{equation}

Thus, $ (p_{i} - p_{Q}) $ can be rewritten as a function of $ p_{k} - p_{Q} $:
\begin{equation} \label{eq:pi_minus_pq}
\begin{split}
(p_{i} - p_{Q}) = \frac{- \xi_{M'L,M}(p_{M} - p_{Q})
						- \xi_{L'M,L}(p_{L} - p_{Q})
						- (\xi_{M'L,k} + \xi_{L'M,k})(p_{k} - p_{Q})
}{\xi_{M'L,i} + \xi_{L'M,i}}
\end{split}
\end{equation}

Performing the same procedure of eqs. \ref{eq:side_flux_L_M_rearranged} - \ref{eq:side_flux_M_L_summation_rearranged} to the face shared by volumes $ M $ and $ W $ yields an expression for $ (p_{j} - p_{Q}) $ as a function of $ (p_{k} - p_{Q}) $:
\begin{equation} \label{eq:pj_minus_pq}
\begin{split}
(p_{j} - p_{Q}) = \frac{- \xi_{M'W,M}(p_{M} - p_{Q})
						- \xi_{W'M,L}(p_{W} - p_{Q})
						- (\xi_{M'W,k} + \xi_{W'M,k})(p_{k} - p_{Q})
}{\xi_{M'W,j} + \xi_{W'M,j}}
\end{split}
\end{equation}

Also, the same procedure of eqs. \ref{eq:side_flux_L_M_rearranged} - \ref{eq:side_flux_M_L_summation_rearranged} is applied now to the face shared by volumes $ M $ and $ R $ and the expression for $ (p_{k} - p_{Q}) $ can be obtained by replacing eqs. \ref{eq:pi_minus_pq} and \ref{eq:pj_minus_pq}:

\begin{equation} \label{eq:pk_minus_pq}
\begin{split}
(p_{k} - p_{Q}) = A_{1}(p_{M}- p_{Q})-A_{2}(p_{W} - p_{Q}) - A_{3}(p_{L} -p_{Q})+A_{4}(p_{R} - p_{Q})
\end{split}
\end{equation}
where
\begin{displaymath}
	A_{1} = \frac{\xi_{M'R,M} -\Bigg[\Bigg(\frac{\xi_{M'R,j}+\xi_{R'M,j}}{\xi_{M'L,j}+\xi_{L'M,j}}\Bigg)\xi_{M'W,M}+\Bigg(\frac{\xi_{M'R,i}+\xi_{R'M,i}}{\xi_{M'L,i}+\xi_{L'M,i}}\Bigg)\xi_{M'L,M}\Bigg]}{\Bigg[\frac{\xi_{M'W,k} + \xi_{W'M,k}}{\xi_{M'W,j} + \xi_{W'M,j}} \Bigg](\xi_{M'R,i} + \xi_{R'M,i}) + \Bigg[ \frac{\xi_{M'L,k} + \xi_{L'M,k}}{\xi_{M'L,i} + \xi_{L'M,i}} \Bigg](\xi_{M'R,j}+\xi_{R'M,j})}
\end{displaymath}
\begin{displaymath}
	A_{2} = \frac{\Bigg(\frac{\xi_{M'R,j}+\xi_{R'M,j}}{\xi_{M'W,j}+\xi_{W'M,j}}\Bigg)\xi_{W'M,W}}{\Bigg[\frac{\xi_{M'W,k} + \xi_{W'M,k}}{\xi_{M'W,j} + \xi_{W'M,j}} \Bigg](\xi_{M'R,i} + \xi_{R'M,i}) + \Bigg[ \frac{\xi_{M'L,k} + \xi_{L'M,k}}{\xi_{M'L,i} + \xi_{L'M,i}} \Bigg](\xi_{M'R,j}+\xi_{R'M,j})}
\end{displaymath}
\begin{displaymath}
	A_{3} = \frac{\Bigg(\frac{\xi_{M'R,i}+\xi_{R'M,i}}{\xi_{M'L,i}+\xi_{L'M,i}}\Bigg)\xi_{L'M,L}}{\Bigg[\frac{\xi_{M'W,k} + \xi_{W'M,k}}{\xi_{M'W,j} + \xi_{W'M,j}} \Bigg](\xi_{M'R,i} + \xi_{R'M,i}) + \Bigg[ \frac{\xi_{M'L,k} + \xi_{L'M,k}}{\xi_{M'L,i} + \xi_{L'M,i}} \Bigg](\xi_{M'R,j}+\xi_{R'M,j})}
\end{displaymath}
\begin{displaymath}
	A_{4} = \frac{\xi_{R'M,R}}{\Bigg[\frac{\xi_{M'W,k} + \xi_{W'M,k}}{\xi_{M'W,j} + \xi_{W'M,j}} \Bigg](\xi_{M'R,i} + \xi_{R'M,i}) + \Bigg[ \frac{\xi_{M'L,k} + \xi_{L'M,k}}{\xi_{M'L,i} + \xi_{L'M,i}} \Bigg](\xi_{M'R,j}+\xi_{R'M,j})}
\end{displaymath}

Now eqs. \ref{eq:pi_minus_pq} and \ref{eq:pj_minus_pq} can be replaced in eq. \ref{eq:flux_m}, and the only unknown is $ p_{k} $:
\begin{equation} \label{eq:flux_M_replacing_i_j}
\begin{split}
\vec{F}_{M} \cdot \vec{N}_{ijk} = \eta_{M,k}(p_{k} - p_{Q}) \\
 - \eta_{M,j} \Bigg[ \frac{- \xi_{M'W,M}(p_{M} - p_{Q}) - \xi_{W'M,W}(p_{W} - p_{Q}) - (\xi_{M'W,k} + \xi_{W'M,k})(p_{k} - p_{Q})}{\xi_{M'W,j} + \xi_{W'M,j}} \Bigg] \\
 - \eta_{M,i} \Bigg[ \frac{- \xi_{M'L,M}(p_{M} - p_{Q})- \xi_{L'M,L}(p_{L} - p_{Q})- (\xi_{M'L,k} + \xi_{L'M,k})(p_{k} - p_{Q})}{\xi_{M'L,i} + \xi_{L'M,i}} \Bigg]
\end{split}
\end{equation}
where:
\begin{displaymath}
	\eta_{M,k} = \frac{\vec{N}^{T}_{ijk} \cdot K_{M} \cdot \vec{N}_{jQi}}{3V_{M}}
\end{displaymath}

\begin{displaymath}
	\eta_{M,j} = \frac{\vec{N}^{T}_{ijk} \cdot K_{M} \cdot \vec{N}_{iQk}}{3V_{M}}
\end{displaymath}

\begin{displaymath}
	\eta_{M,i} = \frac{\vec{N}^{T}_{ijk} \cdot K_{M} \cdot \vec{N}_{kQj}}{3V_{M}}
\end{displaymath}

Thus, rearranging eq. \ref{eq:flux_M_replacing_i_j}:

\hl{must check if equation is fine!!}
\begin{equation} \label{eq:flux_M_replacing_i_j}
\begin{split}
\vec{F}_{M} \cdot \vec{N}_{ijk} = \Bigg[ \frac{(\xi_{M'W,k} + \xi_{W'M,k})}{(\xi_{M'W,j} + \xi_{W'M,j})}\eta_{M,j} + \frac{(\xi_{M'L,k} + \xi_{L'M,k})}{(\xi_{M'L,i} + \xi_{L'M,i})}\eta_{M,i} + \eta_{M,k} \Bigg](p_{k} - p_{Q}) \\
+ \Bigg[\frac{\xi_{M'L,M}}{\xi_{M'L,i} + \xi_{L'M,i}}\eta_{M,i} + \frac{\xi_{M'W,M}}{\xi_{M'W,j} + \xi_{W'M,j}}\eta_{M,j} \Bigg](p_{M} - p_{Q}) \\
+ \frac{\xi_{W'M,W}}{\xi_{M'W,j} + \xi_{W'M,j}}\eta_{M,j}(p_{W} - p_{Q}) 
+\frac{\xi_{L'M,L}}{\xi_{M'L,i} + \xi_{L'M,i}}\eta_{M,i}(p_{L} - p_{Q})
\end{split}
\end{equation}

Now we can replace eq. \ref{eq:pk_minus_pq} in eq. \ref{eq:flux_M_replacing_i_j} to obtain an expression only in terms of the volumes centroids:
\begin{equation} \label{eq:no_i_j_k_terms}
\begin{split}
\vec{F}_{M} \cdot \vec{N}_{ijk} = B_{4}\Bigg[A_{1}(p_{M}- p_{Q})-A_{2}(p_{W} - p_{Q}) - A_{3}(p_{L} -p_{Q})-p_{Q})+A_{4}(p_{R} - p_{Q})\Bigg] \\
+ B_{1}(p_{M} - p_{Q}) + B_{2}(p_{W} - p_{Q})+B_{3}(p_{L} - p_{Q})
\end{split}
\end{equation}

where:

\begin{displaymath}
B_{1} = \Bigg[\frac{\xi_{M'L,M}}{\xi_{M'L,i} + \xi_{L'M,i}}\eta_{M,i} + \frac{\xi_{M'W,M}}{\xi_{M'W,j} + \xi_{W'M,j}}\eta_{M,j} \Bigg]
\end{displaymath}
\begin{displaymath}
B_{2} = \frac{\xi_{W'M,W}}{\xi_{M'W,j} + \xi_{W'M,j}}\eta_{M,j}
\end{displaymath}
\begin{displaymath}
B_{3} = \frac{\xi_{L'M,L}}{\xi_{M'L,i} + \xi_{L'M,i}}\eta_{M,i}
\end{displaymath}
\begin{displaymath}
B_{4} = \Bigg[ \frac{(\xi_{M'W,k} + \xi_{W'M,k})}{(\xi_{M'W,j} + \xi_{W'M,j})}\eta_{M,j} + \frac{(\xi_{M'L,k} + \xi_{L'M,k})}{(\xi_{M'L,i} + \xi_{L'M,i})}\eta_{M,i} -\eta_{M,k} \Bigg]
\end{displaymath}

thus, for the volume $ M $, the flux through the face $ ijk $ is given by:

\begin{equation} \label{eq:flux_M_coefficients}
\begin{split}
\vec{F}_{M} \cdot \vec{N}_{ijk} = \Bigg[A_{1}B_{4}-B_{1}\Bigg] (p_{M}-p_{Q})+ \Bigg[A_{2}B_{4}-B_{2}\Bigg](p_{W} - p_{Q})+\Bigg[A_{3}B_{4}-B_{3}\Bigg](p_{L} - p_{Q})+A_{4}B_{4}(p_{R} - p_{Q})
\end{split}
\end{equation}

In a similar way, the flux to the neighboring volumes to the volume $ M $ can be obtained. performing the procedure from eq. \ref{eq:flux_m} to eq. \ref{eq:flux_M_coefficients}, on each of the neighboring volumes of $ M $, one can obtain the particular s et of all terms dependent on $ p_{M} $. This is done first to volume $ W $, then $ R $ and $ L $, in order to obtain all the terms that will account on the weight of $ p_{M} $:

\begin{equation} \label{eq:flux_volume_W_with_eta}
\vec{F}_{W} \cdot \vec{N}_{rjk} = - \eta_{W,j}(p_{j} - p_{Q}) - \eta_{W,r}(p_{r} - p_{Q}) - \eta_{W,k}(p_{k} - p_{Q})
\end{equation}

The flux through the side shared by volumes $ W $ and $ S $, $ F_{W|S} $ is calculated to give an expression of $ p_{r} - p_{Q} $:

\begin{equation}  \label{eq:pr_minus_pq_W}
p_{r} - p_{Q} = \frac{-\xi_{W'S,W}(p_{W}-p_{Q})-\xi_{S'W,s}(p_{S}-p_{Q}) -(\xi_{W'S,j}+\xi_{S'W,j})(p_{j}-p_{Q})}{\xi_{W'S,r}+\xi_{S'W,r}}
\end{equation}

Similarly, the expression for $ p_{k} - p_{Q} $ can be obtained by the flux through the side shared by volumes $ W $ and $ M $:
\begin{equation}  \label{eq:pk_minus_pq_W}
p_{k} - p_{Q} = \frac{-\xi_{W'M,W}(p_{W}-p_{Q})-\xi_{M'W,M}(p_{M}-p_{Q}) -(\xi_{W'M,j}+\xi_{M'W,j})(p_{j}-p_{Q})}{\xi_{W'M,k}+\xi_{M'W,k}}
\end{equation}

The flux through the side shred by $ L $ and $ M $, when replacing eqs. \ref{eq:pr_minus_pq_W} and \ref{eq:pk_minus_pq_W} yields the expression of $ p_{j} - p_{Q} $:
\begin{equation} \label{eq:pj_minus_pq_W}
\begin{split}
p_{j}-p_{Q} = C_{1}(p_{W}-p_{Q}) - C_{2}(p_{M} - p_{Q}) - C_{3}(p_{S}-p_{Q})+C_{4}(p_{V}-p_{Q})
\end{split}
\end{equation}
where:

\begin{displaymath}
C_{1} = \frac{\xi_{W'V,W}-\Bigg[\frac{(\xi_{W'V,k}+\xi_{V'W,k})}{\xi_{W'M,k}+\xi_{M'W,k}}\xi_{W'M,W}+\frac{(\xi_{W'V,r}+\xi_{V'W,r})}{\xi_{W'S,r}+\xi_{S'W,r}}\xi_{W'S,W}\Bigg]}{\frac{(\xi_{W'M,j}+\xi_{M'W,j})}{\xi_{W'M,k}+\xi_{M'W,k}}(\xi_{W'V,k}+\xi_{V'W,k})+\frac{(\xi_{W'S,j}+\xi_{S'W,j})}{\xi_{W'S,r}+\xi_{S'W,r}}(\xi_{W'V,r}+\xi_{V'W,r})}
\end{displaymath}

\begin{displaymath}
C_{2} =  \frac{\xi_{M'W,M}\Bigg[\frac{(\xi_{W'V,k}+\xi_{V'W,k})}{\xi_{W'M,k}+\xi_{M'W,k}}\Bigg]}{\frac{(\xi_{W'M,j}+\xi_{M'W,j})}{\xi_{W'M,k}+\xi_{M'W,k}}(\xi_{W'V,k}+\xi_{V'W,k})+\frac{(\xi_{W'S,j}+\xi_{S'W,j})}{\xi_{W'S,r}+\xi_{S'W,r}}(\xi_{W'V,r}+\xi_{V'W,r})}
\end{displaymath}

\begin{displaymath}
C_{3} =  \frac{\xi_{S'W,s}\Bigg[\frac{(\xi_{W'V,r}+\xi_{V'W,r})}{\xi_{W'S,r}+\xi_{S'W,r}}\Bigg]}{\frac{(\xi_{W'M,j}+\xi_{M'W,j})}{\xi_{W'M,k}+\xi_{M'W,k}}(\xi_{W'V,k}+\xi_{V'W,k})+\frac{(\xi_{W'S,j}+\xi_{S'W,j})}{\xi_{W'S,r}+\xi_{S'W,r}}(\xi_{W'V,r}+\xi_{V'W,r})}
\end{displaymath}

\begin{displaymath}
C_{4} =  \frac{\xi_{V'W,V}}{\frac{(\xi_{W'M,j}+\xi_{M'W,j})}{\xi_{W'M,k}+\xi_{M'W,k}}(\xi_{W'V,k}+\xi_{V'W,k})+\frac{(\xi_{W'S,j}+\xi_{S'W,j})}{\xi_{W'S,r}+\xi_{S'W,r}}(\xi_{W'V,r}+\xi_{V'W,r})}
\end{displaymath}


Therefore, replacing eqs. \ref{eq:pr_minus_pq_W}, \ref{eq:pk_minus_pq_W} and \ref{eq:pj_minus_pq_W} in eq. \ref{eq:flux_volume_W_with_eta}, we have the expression of the flux in $ W $ for only cell center unknowns:

\begin{equation} \label{eq:flux_W_coefficients}
\begin{split}
\vec{F}_{W} \cdot \vec{N}_{rjk} = \Bigg[C_{1}D_{4}-D_{1}\Bigg](p_{W}-p_{Q})+ \Bigg[C_{2}D_{4}-D_{2}\Bigg](p_{M} - p_{Q})+\\
\Bigg[C_{3}D_{4}-D_{3}\Bigg](p_{S} - p_{Q})+C_{4}D_{4}(p_{V} - p_{Q})
\end{split}
\end{equation}
where:
\begin{displaymath}
D_{1} = \Bigg[\frac{\xi_{W'M,W}}{\xi_{W'M,k}+\xi_{M'W,k}}\eta_{W,k} + \frac{\xi_{W'S,W}}{\xi_{W'S,r}+\xi_{S'W,s}}\eta_{W,r}\Bigg]
\end{displaymath}

\begin{displaymath}
D_{2} = \Bigg[\frac{\xi_{M'W,M}}{\xi_{W'M,k}+\xi_{M'W,k}}\eta_{W,k}\Bigg]
\end{displaymath}

\begin{displaymath}
D_{3} = \Bigg[\frac{\xi_{S'W,s}}{\xi_{W'S,r}+\xi_{S'W,s}}\eta_{W,r}\Bigg]
\end{displaymath}

\begin{displaymath}
D_{4} = \Bigg[\frac{(\xi_{W'M,j}+\xi_{M'W,j})}{\xi_{W'M,k}+\xi_{M'W,k}}\eta_{W,k}+\frac{(\xi_{W'S,j}+\xi_{S'W,j})}{\xi_{W'S,r}+\xi_{S'W,s}}\eta_{W,r}- \eta_{W,j}\Bigg]
\end{displaymath}

Note that eq. \ref{eq:flux_W_coefficients} shows one term that is function of $ p_{M} $. In fact the term $ C_{2}D_{4} - D_{3}$ is due to the flux on the right-sided face of the volume. Therefore, upon the flux for volumes $ R $ and $ L $ there should appear a $ E_{3}F_{4} + F_{3} $ and a $ G_{4}H_{4} $, respectively.

The flux through the face $ ijl $ that is part of the volume $ R $ is given by:

\begin{equation} \label{eq:flux_volume_R_with_eta}
\vec{F}_{R} \cdot \vec{N}_{ijl} = - \eta_{R,i}(p_{i} - p_{Q}) - \eta_{R,l}(p_{l} - p_{Q}) - \eta_{R,j}(p_{j} - p_{Q})
\end{equation}

An expression for $ p_{i} - p_{Q} $ and $ p_{l} - p_{Q} $ is derived by performing the sided flux similar to the previous calculations?
\begin{equation}  \label{eq:pi_minus_pq_R}
p_{i} - p_{Q} = \frac{-\xi_{R'M,R}(p_{R}-p_{Q})-\xi_{M'R,M}(p_{M}-p_{Q}) -(\xi_{R'M,j}+\xi_{M'R,j})(p_{j}-p_{Q})}{\xi_{R'M,i}+\xi_{M'R,i}}
\end{equation}

\begin{equation}  \label{eq:pl_minus_pq_R}
p_{l} - p_{Q} = \frac{-\xi_{R'O,R}(p_{R}-p_{Q})-\xi_{O'R,O}(p_{O}-p_{Q}) -(\xi_{R'O,j}+\xi_{O'R,j})(p_{j}-p_{Q})}{\xi_{R'O,l}+\xi_{O'R,l}}
\end{equation}

\begin{equation} \label{eq:pj_minus_pq_R}
\begin{split}
p_{j}-p_{Q} = E_{1}(p_{R}-p_{Q}) - E_{2}(p_{O} - p_{Q}) - E_{3}(p_{M}-p_{Q})+E_{4}(p_{P}-p_{Q})
\end{split}
\end{equation}
where:

\begin{displaymath}
E_{1} =  \frac{\xi_{R'P,R} - \Bigg[\Bigg(\frac{\xi_{R'P,i}+\xi_{P'R,i}}{\xi_{R'M,i}+\xi_{M'R,i}}\Bigg)\xi_{R'M,R}+\Bigg(\frac{\xi_{R'P,l}+\xi_{P'R,l}}{\xi_{R'O,l}+\xi_{O'R,l}}\Bigg)\xi_{R'O,R}\Bigg]}{(\xi_{R'P,i}+\xi_{P'R,i})\frac{(\xi_{R'M,j}+\xi_{M'R,j})}{\xi_{R'M,i}+\xi_{M'R,i}}
+(\xi_{R'P,l}+\xi_{P'R,l})\frac{(\xi_{R'O,j}+\xi_{O'R,j})}{\xi_{R'O,l}+\xi_{O'R,l}}}
\end{displaymath}

\begin{displaymath}
E_{2} = \frac{\xi_{O'R,O}\Bigg[\frac{\xi_{R'P,l}+\xi_{P'R,l}}{\xi_{R'O,l}+\xi_{O'R,l}}\Bigg]}{(\xi_{R'P,i}+\xi_{P'R,i})\frac{(\xi_{R'M,j}+\xi_{M'R,j})}{\xi_{R'M,i}+\xi_{M'R,i}}
+(\xi_{R'P,l}+\xi_{P'R,l})\frac{(\xi_{R'O,j}+\xi_{O'R,j})}{\xi_{R'O,l}+\xi_{O'R,l}}}
\end{displaymath}

\begin{displaymath}
E_{3} = \frac{\xi_{M'R,M}\Bigg[\frac{\xi_{R'P,i}+\xi_{P'R,i}}{\xi_{R'M,i}+\xi_{M'R,i}}\Bigg]}{(\xi_{R'P,i}+\xi_{P'R,i})\frac{(\xi_{R'M,j}+\xi_{M'R,j})}{\xi_{R'M,i}+\xi_{M'R,i}}
+(\xi_{R'P,l}+\xi_{P'R,l})\frac{(\xi_{R'O,j}+\xi_{O'R,j})}{\xi_{R'O,l}+\xi_{O'R,l}}}
\end{displaymath}

\begin{displaymath}
E_{4} = \frac{\xi_{P'R,P}}{(\xi_{R'P,i}+\xi_{P'R,i})\frac{(\xi_{R'M,j}+\xi_{M'R,j})}{\xi_{R'M,i}+\xi_{M'R,i}}
+(\xi_{R'P,l}+\xi_{P'R,l})\frac{(\xi_{R'O,j}+\xi_{O'R,j})}{\xi_{R'O,l}+\xi_{O'R,l}}}
\end{displaymath}

\begin{equation} \label{eq:flux_R_coefficients}
\begin{split}
\vec{F}_{R} \cdot \vec{N}_{ijl} = \Bigg[E_{1}F_{4}-F_{1}\Bigg] (p_{R}-p_{Q})+ \Bigg[E_{2}F_{4}-F_{2}\Bigg](p_{O} - p_{Q})+\Bigg[E_{3}F_{4}-F_{3}\Bigg](p_{M} - p_{Q})+E_{4}F_{4}(p_{P} - p_{Q})
\end{split}
\end{equation}
where:
\begin{displaymath}
F_{1} = \Bigg[\frac{\xi_{R'M,R}}{\xi_{R'M,i}+\xi_{M'R,i}}\eta_{R,i}+ \frac{\xi_{R'O,R}}{\xi_{R'O,l}+\xi_{O'R,l}}\eta_{R,l}\Bigg]
\end{displaymath}

\begin{displaymath}
F_{2} = \Bigg[\frac{\xi_{O'R,O}}{\xi_{R'O,l}+\xi_{O'R,l}}\eta_{R,l}\Bigg]
\end{displaymath}

\begin{displaymath}
F_{3} = \Bigg[\frac{\xi_{M'R,M}}{\xi_{R'M,i}+\xi_{M'R,i}}\eta_{R,i}\Bigg]
\end{displaymath}

\begin{displaymath}
F_{4} = \Bigg[\frac{(\xi_{R'M,j}+\xi_{M'R,j})}{\xi_{R'M,i}+\xi_{M'R,i}}\eta_{R,i}
+\frac{(\xi_{R'O,j}+\xi_{O'R,j})}{\xi_{R'O,l}+\xi_{O'R,l}} \eta_{R,l}-\eta_{R,j}\Bigg]
\end{displaymath}

Finally, equations for volume $ L $ can be written:

\begin{equation} \label{eq:flux_volume_L_with_eta}
\vec{F}_{L} \cdot \vec{N}_{iwk} = - \eta_{L,k}(p_{k} - p_{Q}) - \eta_{L,i}(p_{i} - p_{Q}) - \eta_{L,w}(p_{w} - p_{Q})
\end{equation}

\begin{equation} \label{eq:pi_minus_pq_L}
p_{i} - p_{Q} = \frac{-\xi_{L'N,L}(p_{L}-p_{Q})-\xi_{N'L,N}(p_{N}-p_{Q}) -(\xi_{L'N,w}+\xi_{N'L,w})(p_{w}-p_{Q})}{\xi_{L'N,i}+\xi_{N'L,i}}
\end{equation}


\begin{equation} \label{eq:pk_minus_pq_L}
p_{k} - p_{Q} = \frac{-\xi_{L'H,L}(p_{L}-p_{Q})-\xi_{H'L,H}(p_{H}-p_{Q}) -(\xi_{L'H,w}+\xi_{H'L,w})(p_{w}-p_{Q})}{\xi_{L'H,k}+\xi_{H'L,k}}
\end{equation}

\begin{equation} \label{eq:pw_minus_pq_L}
\begin{split}
p_{w}-p_{Q} = G_{1}(p_{L}-p_{Q}) - G_{2}(p_{H} - p_{Q}) - G_{3}(p_{N}-p_{Q})+G_{4}(p_{M}-p_{Q})
\end{split}
\end{equation}
where:

\begin{displaymath}
G_{1} = \frac{\xi_{L'M,L} - \Bigg[\xi_{L'N,L}\frac{(\xi_{L'M,i}+\xi_{M'L,i})}{\xi_{L'N,i}+\xi_{N'L,i}} + \xi_{L'H,L}\frac{(\xi_{L'M,k}+\xi_{M'L,k})}{\xi_{L'H,k}+\xi_{H'L,k}} \Bigg]}{\Bigg[ (\xi_{L'M,i}+\xi_{M'L,i})\frac{(\xi_{L'N,w}+\xi_{N'L,w})}{\xi_{L'N,i}+\xi_{N'L,i}} + (\xi_{L'M,k}+\xi_{M'L,k})	\frac{(\xi_{L'H,w}+\xi_{H'L,w})}{\xi_{L'H,k}+\xi_{H'L,k}} \Bigg]}
\end{displaymath}

\begin{displaymath}
G_{2} =\frac{\xi_{H'L,H}\Bigg[ \frac{(\xi_{L'M,k}+\xi_{M'L,k})}{\xi_{L'H,k}+\xi_{H'L,k}} \Bigg]}{\Bigg[ (\xi_{L'M,i}+\xi_{M'L,i})\frac{(\xi_{L'N,w}+\xi_{N'L,w})}{\xi_{L'N,i}+\xi_{N'L,i}} + (\xi_{L'M,k}+\xi_{M'L,k})	\frac{(\xi_{L'H,w}+\xi_{H'L,w})}{\xi_{L'H,k}+\xi_{H'L,k}} \Bigg]}
\end{displaymath}

\begin{displaymath}
G_{3} = \frac{\xi_{N'L,N}\Bigg[ \frac{(\xi_{L'M,i}+\xi_{M'L,i})}{\xi_{L'N,i}+\xi_{N'L,i}}\Bigg]}{\Bigg[ (\xi_{L'M,i}+\xi_{M'L,i})\frac{(\xi_{L'N,w}+\xi_{N'L,w})}{\xi_{L'N,i}+\xi_{N'L,i}} + (\xi_{L'M,k}+\xi_{M'L,k})	\frac{(\xi_{L'H,w}+\xi_{H'L,w})}{\xi_{L'H,k}+\xi_{H'L,k}} \Bigg]}
\end{displaymath}

\begin{displaymath}
G_{4} = \frac{\xi_{M'L,M}}{\Bigg[ (\xi_{L'M,i}+\xi_{M'L,i})\frac{(\xi_{L'N,w}+\xi_{N'L,w})}{\xi_{L'N,i}+\xi_{N'L,i}} + (\xi_{L'M,k}+\xi_{M'L,k})	\frac{(\xi_{L'H,w}+\xi_{H'L,w})}{\xi_{L'H,k}+\xi_{H'L,k}} \Bigg]}
\end{displaymath}

\begin{equation} \label{eq:flux_L_coefficients}
\begin{split}
\vec{F}_{L} \cdot \vec{N}_{iwk} = \Bigg[G_{1}H_{4}-H_{1}\Bigg] (p_{L}-p_{Q})+ \Bigg[G_{2}H_{4}-H_{2}\Bigg](p_{H} - p_{Q})+\Bigg[G_{3}H_{4}-H_{3}\Bigg](p_{N} - p_{Q})+G_{4}H_{4}(p_{M} - p_{Q})
\end{split}
\end{equation}

where:

\begin{displaymath}
H_{1} = \eta_{L,i}\frac{\xi_{L'N,L}}{\xi_{L'N,i}+\xi_{N'L,i}}+\eta_{L,k}\frac{\xi_{L'H,L}}{\xi_{L'H,k}+\xi_{H'L,k}}
\end{displaymath}

\begin{displaymath}
H_{2} = \frac{\xi_{H'L,H}}{\xi_{L'H,k}+\xi_{H'L,k}}\eta_{L,k}
\end{displaymath}

\begin{displaymath}
H_{3} = \frac{\xi_{N'L,N}}{\xi_{L'N,i}+\xi_{N'L,i}}\eta_{L,i}
\end{displaymath}

\begin{displaymath}
H_{4} = \Bigg[\frac{ (\xi_{L'H,w}+\xi_{H'L,w})}{\xi_{L'H,k}+\xi_{H'L,k}}\eta_{L,k}
 +\frac{(\xi_{L'N,w}+\xi_{N'L,w})}{\xi_{L'N,i}+\xi_{N'L,i}}\eta_{L,i} - \eta_{L,w}\Bigg]
\end{displaymath}

Now, suppose that there are N volumes that share the vertex $ Q $. Then the summation of the flux through this control volume around the vertex $ Q $ should yield:
\begin{equation} \label{eq:summ_all_surrounding_Q}
\sum_{n = 1}^{N}\vec{F}_{n} \cdot \vec{N}_{n}=0
\end{equation}

\begin{equation} \label{eq:summ_all_surrounding_Q_coeffic}
\sum_{n=1}^{N}[A_{1}B_{4}+C_{2}D_{4}+E_{3}F_{4}+G_{4}H_{4}-(B_{1}+D_{2}+F_{3})]_{n}(p_{n}-p_{Q})=0
\end{equation}

\begin{equation} \label{eq:summ_all_surrounding_isolation_pq}
p_{Q}=\frac{\sum_{n=1}^{N}[A_{1}B_{4}+C_{2}D_{4}+E_{3}F_{4}+G_{4}H_{4}-(B_{1}+D_{2}+F_{3})]_{n}p_{n}}{\sum_{n=1}^{N}[A_{1}B_{4}+C_{2}D_{4}+E_{3}F_{4}+G_{4}H_{4}-(B_{1}+D_{2}+F_{3})]_{n}}
\end{equation}
where for every $ n $ volume, there will be a left sided volume $ L $, a right sided volume $ R $ and a just below (or above) volume $ W $. Note that sometimes each $ L $, $ R $ and $ W $ dependent expressions must depend on their own vicinity. With that, we can generalize the expression for each term on eq. \ref{eq:summ_all_surrounding_isolation_pq}:

\begin{displaymath}
	A_{1} = \frac{\xi_{n'R,n} -\Bigg[\Bigg(\frac{\xi_{n'R,j}+\xi_{R'n,j}}{\xi_{n'L,i}+\xi_{L'n,i}}\Bigg)\xi_{n'W,n}+\Bigg(\frac{\xi_{n'R,i}+\xi_{R'n,i}}{\xi_{n'L,i}+\xi_{L'n,i}}\Bigg)\xi_{n'L,i}\Bigg]}{\Bigg[\frac{\xi_{n'R,j} + \xi_{R'n,j}}{\xi_{n'L,j} + \xi_{L'n,j}} \Bigg](\xi_{n'W,k} + \xi_{W'n,k}) + \Bigg[ \frac{\xi_{n'R,i} + \xi_{R'n,i}}{\xi_{n'L,i} + \xi_{L'n,i}} \Bigg](\xi_{n'L,k}+\xi_{L'n,k})}
\end{displaymath}

\begin{displaymath}
B_{1} = \Bigg[\frac{\xi_{n'L,n}}{\xi_{n'L,i} + \xi_{L'n,i}}\eta_{n,i} + \frac{\xi_{n'W,n}}{\xi_{n'W,j} + \xi_{W'n,j}}\eta_{n,j} \Bigg]
\end{displaymath}

\begin{displaymath}
B_{4} = \Bigg[ \frac{(\xi_{n'W,k} + \xi_{W'n,k})}{(\xi_{n'W,j} + \xi_{W'n,j})}\eta_{n,j} + \frac{(\xi_{n'L,k} + \xi_{L'n,k})}{(\xi_{n'L,i} + \xi_{L'n,i})}\eta_{n,i} -\eta_{n,k} \Bigg]
\end{displaymath}

\begin{displaymath}
C_{2} =  \frac{\xi_{n'W,n}\Bigg[\frac{(\xi_{W'V,k}+\xi_{V'W,k})}{\xi_{W'n,k}+\xi_{n'W,k}}\Bigg]}{\frac{(\xi_{W'n,j}+\xi_{n'W,j})}{\xi_{W'n,k}+\xi_{n'W,k}}(\xi_{W'V,k}+\xi_{V'W,k})+\frac{(\xi_{W'S,j}+\xi_{S'W,j})}{\xi_{W'S,r}+\xi_{S'W,r}}(\xi_{W'V,r}+\xi_{V'W,r})}
\end{displaymath}

\begin{displaymath}
D_{2} = \Bigg[\frac{\xi_{n'W,n}}{\xi_{W'n,k}+\xi_{n'W,k}}\eta_{W,k}\Bigg]
\end{displaymath}

\begin{displaymath}
D_{4} = \Bigg[\frac{(\xi_{W'n,j}+\xi_{n'W,j})}{\xi_{W'n,k}+\xi_{n'W,k}}\eta_{W,k}+\frac{(\xi_{W'S,j}+\xi_{S'W,j})}{\xi_{W'S,r}+\xi_{S'W,s}}\eta_{W,r}- \eta_{W,j}\Bigg]
\end{displaymath}

\begin{displaymath}
E_{3} = \frac{\xi_{n'R,n}\Bigg[\frac{\xi_{R'P,i}+\xi_{P'R,i}}{\xi_{R'n,i}+\xi_{n'R,i}}\Bigg]}{(\xi_{R'P,i}+\xi_{P'R,i})\frac{(\xi_{R'n,j}+\xi_{n'R,j})}{\xi_{R'n,i}+\xi_{n'R,i}}
+(\xi_{R'P,l}+\xi_{P'R,l})\frac{(\xi_{R'O,j}+\xi_{O'R,j})}{\xi_{R'O,l}+\xi_{O'R,l}}}
\end{displaymath}

\begin{displaymath}
F_{3} = \Bigg[\frac{\xi_{n'R,n}}{\xi_{R'n,i}+\xi_{n'R,i}}\eta_{R,i}\Bigg]
\end{displaymath}

\begin{displaymath}
F_{4} = \Bigg[\frac{(\xi_{R'n,j}+\xi_{n'R,j})}{\xi_{R'n,i}+\xi_{n'R,i}}\eta_{R,i}
+\frac{(\xi_{R'O,j}+\xi_{O'R,j})}{\xi_{R'O,l}+\xi_{O'R,l}} \eta_{R,l}-\eta_{R,j}\Bigg]
\end{displaymath}

\begin{displaymath}
G_{4} = \frac{\xi_{n'L,n}}{\Bigg[ (\xi_{L'n,i}+\xi_{n'L,i})\frac{(\xi_{L'N,w}+\xi_{N'L,w})}{\xi_{L'N,i}+\xi_{N'L,i}} + (\xi_{L'n,k}+\xi_{n'L,k})	\frac{(\xi_{L'H,w}+\xi_{H'L,w})}{\xi_{L'H,k}+\xi_{H'L,k}} \Bigg]}
\end{displaymath}

\begin{displaymath}
H_{4} = \Bigg[\frac{ (\xi_{L'H,w}+\xi_{H'L,w})}{\xi_{L'H,k}+\xi_{H'L,k}}\eta_{L,k}
 +\frac{(\xi_{L'N,w}+\xi_{N'L,w})}{\xi_{L'N,i}+\xi_{N'L,i}}\eta_{L,i} - \eta_{L,w}\Bigg]
\end{displaymath}

It is easy to note taht eq. \ref{eq:summ_all_surrounding_isolation_pq} is just the same as the proposed weighted eq. \ref{eq:summ_all_omega}, therefore, a relation for the vertex unknowns as a function of all cell centroids is obtained.	
	\subsection{Convergence, Stability and Consistency}
		\subsubsection{Convergence}
		
		\subsubsection{Stability}
		
		\subsubsection{Consistency}

\section{Numerical Results}	
	\begin{equation}
		\begin{split}
 \Bigg(K_{L}^{n}\alpha_{L,Q_{0}i} - \frac{1}{|\vec{N}_{iwk}|}K_{L}^{Q_{0}i}+\frac{2K_{O}^{n}}{h_{O}}+K_{W}^{n}\alpha_{W,Q_{0}r} - \frac{1}{|\vec{N}_{rjk}|}K_{W}^{Q_{0}r}+\frac{2K_{M}^{n}}{h_{M}}\Bigg)(p_{Q_{0}} - p_{k})
\\+
	\Bigg(K_{R}^{n}\alpha_{R,Q_{0}l} - \frac{1}{|\vec{N}_{ijl}|}K_{R}^{Q_{0}l}+K_{N}^{n}\alpha_{N,Q_{0}w} - \frac{1}{|\vec{N}_{ilw}|}K_{N}^{Q_{0}w} \\+ K_{M}^{n}\alpha_{M,Q_{0}j} - \frac{1}{|\vec{N}_{ikj}|}K_{M}^{Q_{0}j}+K_{L}^{n}\alpha_{L,Q_{0}k} - \frac{1}{|\vec{N}_{iwk}|}K_{L}^{Q_{0}k}\Bigg)(p_{Q_{0}} - p_{i})
\\+
\Bigg(K_{R}^{n}\alpha_{R,Q_{0}i} - \frac{1}{|\vec{N}_{ijl}|}K_{R}^{Q_{0}i}+\frac{2K_{K}^{n}}{h_{K}}+K_{V}^{n}\alpha_{V,Q_{0}r} - \frac{1}{|\vec{N}_{rwl}|}K_{V}^{Q_{0}r}+\frac{2K_{N}^{n}}{h_{N}}\Bigg)(p_{Q_{0}} - p_{l})
\\+
\Bigg(K_{K}^{n}\alpha_{K,Q_{0}j} - \frac{1}{|\vec{N}_{rlj}|}K_{K}^{Q_{0}j}+K_{V}^{n}\alpha_{V,Q_{0}l} - \frac{1}{|\vec{N}_{rwl}|}K_{V}^{Q_{0}l}+ \\ K_{O}^{n}\alpha_{O,Q_{0}w} - \frac{1}{|\vec{N}_{rkw}|}K_{O}^{Q_{0}w}+K_{W}^{n}\alpha_{W,Q_{0}k} - \frac{1}{|\vec{N}_{rjk}|}K_{W}^{Q_{0}k}\Bigg)(p_{Q_{0}} - p_{r})
\\+
		\Bigg(K_{K}^{n}\alpha_{K,Q_{0}r} - \frac{1}{|\vec{N}_{rlj}|}K_{K}^{Q_{0}r}+K_{M}^{n}\alpha_{M,Q_{0}i} - \frac{1}{|\vec{N}_{ikj}|}K_{M}^{Q_{0}i}+\frac{2K_{W}^{n}}{h_{W}}+\frac{2K_{R}^{n}}{h_{R}}\Bigg)(p_{Q_{0}} - p_{j})
\\+
		\Bigg(K_{N}^{n}\alpha_{N,Q_{0}i} - \frac{1}{|\vec{N}_{ilw}|}K_{N}^{Q_{0}i} + K_{O}^{n}\alpha_{O,Q_{0}r} - \frac{1}{|\vec{N}_{rkw}|}K_{O}^{Q_{0}r}+\frac{2K_{L}^{n}}{h_{L}}+\frac{2K_{V}^{n}}{h_{V}}\Bigg)(p_{Q_{0}} - p_{w})
\approx 0
		\end{split}
		\end{equation}
		
		\begin{equation}
		\begin{split}
\vec{F}_{M|R} \approx  -\Bigg[ \big( p_{Q_{0}} - p_{i}\big) \Bigg( \beta_{M,Q_{0}j}\frac{K^n_{M}}{h_{O_{M}}} - \frac{1}{|\vec{N}_{Q_{0}ij}|}K^{Q_{0}j}_{M}\Bigg) + \\
\big( p_{Q_{0}} - p_{j} \big) \Bigg(\beta_{M,Q_{0}i}\frac{K^n_{M}}{h_{O_{M}}} - \frac{1}{|\vec{N}_{Q_{0}ij}|}K^{Q_{0}i}_{M}\Bigg)  + 
2 \big(p_{Q_{0}} - p_{M}\big) \frac{K^n_{M}}{h_{O_{M}}} \Bigg]
\end{split}
		\end{equation}
		\begin{equation}
		\begin{split}
\vec{F}_{R|M} \approx  \Bigg[ \big( p_{Q_{0}} - p_{i}\big) \Bigg( \beta_{R,Q_{0}j}\frac{K^n_{R}}{h_{O_{R}}} - \frac{1}{|\vec{N}_{Q_{0}ij}|}K^{Q_{0}j}_{R}\Bigg) + \\
\big( p_{Q_{0}} - p_{j} \big) \Bigg(\beta_{R,Q_{0}i}\frac{K^n_{R}}{h_{O_{R}}} - \frac{1}{|\vec{N}_{Q_{0}ij}|}K^{Q_{0}i}_{R}\Bigg)  + 
2 \big(p_{Q_{0}} - p_{R}\big) \frac{K^n_{R}}{h_{O_{R}}} \Bigg]
\end{split}
		\end{equation}
		
		\begin{equation}
		\vec{F}_{M|R} + \vec{F}_{R|M} \approx 0
		\end{equation}
		
		\begin{equation}
		\begin{split}
\big( p_{Q_{0}} - p_{i}\big) \approx -\frac{
\big( p_{Q_{0}} - p_{j} \big) \Bigg(\beta_{R,Q_{0}i}\frac{K^n_{R}}{h_{O_{R}}} - \frac{1}{|\vec{N}_{Q_{0}ij}|}K^{Q_{0}i}_{R}+\beta_{M,Q_{0}i}\frac{K^n_{M}}{h_{O_{M}}} - \frac{1}{|\vec{N}_{Q_{0}ij}|}K^{Q_{0}i}_{M}\Bigg)  + 
2 \big(p_{Q_{0}} - p_{M}\big) \frac{K^n_{M}}{h_{O_{M}}} +2 \big(p_{Q_{0}} - p_{R}\big) \frac{K^n_{R}}{h_{O_{R}}}}{\Bigg(  \beta_{R,Q_{0}j}\frac{K^n_{R}}{h_{O_{R}}} - \frac{1}{|\vec{N}_{Q_{0}ij}|}K^{Q_{0}j}_{R} + \beta_{M,Q_{0}j}\frac{K^n_{M}}{h_{O_{M}}} - \frac{1}{|\vec{N}_{Q_{0}ij}|}K^{Q_{0}j}_{M}\Bigg)}
\end{split}
		\end{equation}	
\section{Conclusions}

\section*{Acknowledgments}

\paragraph{References}

\end{document}
